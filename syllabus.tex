\documentclass[12pt]{article}
\pagestyle{plain}
\usepackage[pdftex]{graphicx}
\usepackage{fullpage,graphicx,psfrag,url,caption, authblk, amsmath, float, fancyhdr, multicol, hyperref}
\usepackage{amsmath,amsfonts,amssymb}

\fancypagestyle{pages}{
	%Headers
	\fancyhead[L]{Physics 7A, Summer 2024 \\ Section 1xx}
	%\fancyhead[C]{\thepage}
	\fancyhead[R]{Discussion 1 \\ April 23, 2024}
\renewcommand{\headrulewidth}{0pt}
	%Footers
	%\fancyfoot[L]{}
	\fancyfoot[C]{\thepage}
	%\fancyfoot[R]{}
\renewcommand{\footrulewidth}{0.0pt}
}
\headheight=0pt
\footskip=0pt

\setlength{\oddsidemargin}{0 in}
\setlength{\evensidemargin}{0 in}
\setlength{\topmargin}{-0.25 in}
\setlength{\textwidth}{6.5 in}
\setlength{\textheight}{8.5 in}
\setlength{\headsep}{1 in}
\setlength{\parindent}{0 in}
\setlength{\parskip}{0.1 in}

\begin{document}

\pagestyle{pages}



\begin{center}
\vspace{3in}
{\Large Discussion 1 } \\[0.05in]
Vector Algebra, Kinematics \\ [0.5in]
\end{center}

% this is a comment

\begin{enumerate}

\item This is my solution to question 1 
	\begin{enumerate} 
		\item This is my solution to question 1a
		\item The DTFT equation is $X(e^{j\omega}) =\sum_{k=-\infty}^\infty
			 x[k]e^{-j\omega k}$
		\item The IDTFT equation is $\frac{1}{2\pi}\int_{-\pi}^\pi
			 X(e^{j\omega})e^{j\omega n} d\omega$
	\end{enumerate}

\pagebreak 

Here's how to write verbatim:
\begin{verbatim}
> disp('This is some Matlab Code')
> grade = 10;
> 
\end{verbatim}

Now, here's a figure:
\begin{center}
		%\includegraphics[width=1 \textwidth]{figure.pdf}
\end{center}
\captionof{figure}{This is the caption}
\end{enumerate}

\pagebreak

\begin{table}[h]
\begin{tabular}{ccccc}
            & Point 1 & Point 2 & Point 3 & Point 4 \\
Height (mm) & 13      & 45      & 46      & 19      \\
Temperature ($^oC$) & 0       & 48      & 48      & 0       \\
Mass (g)        & 50     & 50     & 0      & 0      \\
Time (s)    & 0       & 25      & 40      & 60      \\
Height      & 14      & 40      & 42      & 14      \\
Temperature ($^oC$) & 0       & 48      & 48      & 0       \\
Mass (g)       & 100    & 100    & 0      & 0      \\
Time (s)    & 0       & 25      & 35      & 50      \\
\end{tabular}
\end{table}

\end{document}
