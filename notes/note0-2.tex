\documentclass[11pt]{article}
\usepackage[pdftex]{graphicx}
\usepackage[explicit]{titlesec}
\usepackage[OT1]{fontenc}
\usepackage[most]{tcolorbox}
\usepackage[colorlinks=true, urlcolor=cyan, hyperfootnotes=false]{hyperref}
\usepackage{fullpage, graphicx, psfrag, url, caption, authblk, amsmath, amsfonts, amssymb, float, fancyhdr, multicol, cmbright, xcolor, amsthm, gensymb, physics}

\fancypagestyle{pages}{
	%Headers
	\fancyhead[L]{Physics 7A, Summer 2024 \\ Section 103}
	%\fancyhead[C]{\thepage}
	\fancyhead[R]{Note 0-2}
\renewcommand{\headrulewidth}{0pt}
	%Footers
	%\fancyfoot[L]{}
	\fancyfoot[C]{}
	\fancyfoot[R]{\thepage}
\renewcommand{\footrulewidth}{0pt}
}

\newcommand\blfootnote[1]{
    \begingroup
    \renewcommand\thefootnote{}\footnote{#1}
    \addtocounter{footnote}{-1}
    \endgroup
}

\newcommand{\fig}[4]{
    \begin{figure}[H]
        \centering
        \includegraphics[scale={#3}, angle={#4}]{#1}
        \caption{#2}
        \label{exp4fit}
    \end{figure}
}

\newtheoremstyle{gangnamstyle}{}{}{}{}{\sffamily\bfseries}{.}{ }{}
\tcolorboxenvironment{definition}{boxrule=0pt,boxsep=0pt,colback={blue!10},left=8pt,right=8pt,enhanced jigsaw, borderline west={2pt}{0pt}{blue},sharp corners,before skip=10pt,after skip=10pt,breakable}
\tcolorboxenvironment{example}{boxrule=0pt,boxsep=0pt,colback={orange!10},left=8pt,right=8pt,enhanced jigsaw, borderline west={2pt}{0pt}{orange},sharp corners,before skip=10pt,after skip=10pt,breakable}
\tcolorboxenvironment{problem}{boxrule=0pt,boxsep=0pt,colback={cyan!10},left=8pt,right=8pt,enhanced jigsaw, borderline west={2pt}{0pt}{cyan},sharp corners,before skip=10pt,after skip=10pt,breakable}
\theoremstyle{gangnamstyle}{\newtheorem{definition}{Definition}[]}
\theoremstyle{gangnamstyle}{\newtheorem{example}{Example}[]}
\theoremstyle{gangnamstyle}{\newtheorem{problem}{Problem}[]}

\headheight=0pt
\footskip=0pt
\setlength{\oddsidemargin}{0 in}
\setlength{\evensidemargin}{0 in}
\setlength{\topmargin}{-0.5 in}
\setlength{\textwidth}{6.5 in}
\setlength{\textheight}{8.5 in}
\setlength{\headsep}{0.75 in}
\setlength{\parindent}{0 in}
\setlength{\parskip}{0.1 in}

\begin{document}
\normalfont
\pagestyle{pages}

% Begin Document

\begin{center}
\vspace{3in}
{\Large Note 0, Part 2 } \\[0.05in]
Vector Calculus  \\ 
\blfootnote{If you found any errors, or have any questions about these notes, please contact Jinsheng Li.} \blfootnote{Email Address: \href{mailto:ljs233233@berkeley.edu}{ljs233233@berkeley.edu}} \\ [-0.5in]
\end{center}

\section*{Motivation}

These supplementary notes are written to either discuss a topic we learned in greater detail, or to present an application of the concepts from the class. You are certainly not responsible for the materials discussed here, unless it is also mentioned by the professor in lecture. 

\textit{So why should I still read this?} \\
\textbf{Vector Calculus} will be an integral part of every physics course to come. It is a convenient and heavily used mathematical tool to express vectorial quantities, regardless of their physical nature. Within these sets of notes, we will also discuss the immediate applications of Vector Calculus in Conservative Forces and Potential. 

\subsection*{Review}

You should be familiar with the following relationships that we will use in this note. 

\begin{itemize}
\item Some common derivatives:
\[ \frac{d}{dx}(x^n) = nx^{n - 1} \]
\[ \frac{d}{dx}(e^{ax}) = ae^{ax} \]
\[ \frac{d}{dx}(\ln(x)) = \frac{1}{x} \]
\item Work done by a Force:
\[ W = \int \Vec{F} \cdot d\Vec{r} \]
\item Conservative Forces and Potential in 1-Dimension:
\[ \Vec{F} = - \frac{dU}{dr}\hat{r} \]
\[ U = - \int_{r_0}^r \Vec{F} \cdot d\Vec{r} \]
\item Note: In these notes, we will use the dot on top of a variable refers to the time derivative of the variable. For example: 
\[ v = \Dot{x} = \frac{dx}{dt} \]
\end{itemize}
\pagebreak

\section{Functions of Multiple Variables}
\subsection{Partial Derivatives}

Recall that for a function of one variable, $f(x)$, it's derivative is defined as
\[ \frac{df}{dx} = f'(x) = \lim_{h \to 0} \frac{f(x + h) - f(x)}{h} \]
For a function of multiple variables, $f(x, y)$, we define its partial derivative with respect to one variable (say, $x$) as the rate of change of $f$ with respect to $x$ while keeping $y$ fixed at some constant $y = b$. \\
Notation-wise, instead of $d$, we use the partial symbol $\partial$ in the fraction form ($\frac{\partial f}{\partial x}$ is pronounced "Partial $f$ Partial $x$"). And instead of the prime notation $f'$, we use the subscript $f_x$ to denote the partial derivative of $f$ with respect to x. \\
Just like the derivative of single-variable functions, partial derivatives are also defined in the form of a limit. \\
For a function of two variables 
\[ f = f(x, y) \]
Its partial derivates are defined as 
\[ \frac{\partial f}{\partial x} = f_x(x, y) = \lim_{h \to 0} \frac{f(x + h, y) - f(x, y)}{h} \]
\[ \frac{\partial f}{\partial y} = f_y(x, y) = \lim_{h \to 0} \frac{f(x, y + h) - f(x, y)}{h} \]

\fig{figs/n0/partial.png}{The rate of change $\frac{\partial f}{\partial x}$ when $y = 1$ is held constant.}{0.7}{0}

\pagebreak

Formally, we introduce this definition for the partial derivative. 
\begin{definition}
The Partial Derivative. \\
Give a function of $n$-variables, 
\begin{equation}
f = f(x_1, x_2, ..., x_i, ..., x_n)
\end{equation}
It's partial derivative with respect to the variable $x_i$ is defined as 
\begin{equation}
\begin{aligned}
\frac{\partial f}{\partial x_i} &= f_{x_i}(x_1, x_2, ..., x_i, ..., x_n) \\ 
&= \lim_{h \to 0} \frac{f(x_1, x_2, ..., x_i + h, ..., x_n) - f(x_1, x_2, ..., x_i, ..., x_n)}{h}
\end{aligned}
\end{equation}
In fact, this definition works for any number of variables ($n \geq 1$). The partial derivative of a single-variable function is equivalent to its ordinary derivative. 
\end{definition}

Practically speaking, when taking a partial derivative, all the rules and tricks from the ordinary derivative still apply. We just need to treat all other variables as constants. 

\begin{example}
The Partial Derivative \\
\textit{Stewart, Calculus: Early Transcendentals, Example 14.3.1} \\
If $f(x, y) = x^3 + x^2 y^3 - 2y^2$, find $f_x(2, 1)$ and $f_y(2, 1)$. \\

\textbf{Solution:} \\
Holding $y$ as a constant and differentiating with respect to $x$, 
\[ f_x(x, y) = 3x^2 + 2xy^3 \]
Thus,
\[ f_x(2, 1) = 16 \]
Holding $x$ as a constant and differentiating with respect to $y$, 
\[ f_y(x, y) = 3x^2y^2 - 4y \]
Thus,
\[ f_y(2, 1) = 8 \]
\end{example}

\begin{example}
Implicit Partial Derivatives \\
\textit{Stewart, Calculus: Early Transcendentals, Example 14.3.2} \\
Find $\frac{\partial z}{\partial x}$ and $\frac{\partial z}{\partial y}$ if $z$ is defined implicitly as a function of $x$ and $y$ by the equation
\[ x^3 + y^3 + z^3 + 6xyz = 1 \]

\textbf{Solution:} \\
Differentiate the equation with respect to $x$, while holding $y$ as a constant: 
\[ 3x^2 + 3z^2\frac{\partial z}{\partial x} + 6yz + 6xy\frac{\partial z}{\partial x} = 0 \]
Solving this equation for $\frac{\partial z}{\partial x}$: 
\[ \frac{\partial z}{\partial x} = - \frac{x^2 + 2yz}{z^2 + 2xy} \]
Differentiate the equation with respect to $y$, while holding $x$ as a constant: 
\[ 3y^2 + 3z^2\frac{\partial z}{\partial y} + 6xz + 6xy\frac{\partial z}{\partial y} = 0 \]
Solving this equation for $\frac{\partial z}{\partial y}$: 
\[ \frac{\partial z}{\partial y} = - \frac{y^2 + 2xz}{z^2 + 2xy} \]
\end{example}

\begin{definition}
The Chain Rule for Multivariable Functions. \\
Suppose that $f = f(x, y, z)$ is a differentiable function of $x$, $y$, and $z$, where $x = x(t)$, $y = y(t)$, $z = z(t)$ are differentiable functions of $t$.
Then $f$ is a differentiable function of $t$ and
\begin{equation}
\frac{df}{dt} = \frac{\partial f}{\partial x} \frac{dx}{dt} + \frac{\partial f}{\partial y}\frac{dy}{dt} + \frac{\partial f}{\partial z} \frac{dz}{dt}
\end{equation}
Or, 
\begin{equation}
\Dot{f} = \frac{\partial f}{\partial x} \Dot{x} + \frac{\partial f}{\partial y} \Dot{y} + \frac{\partial f}{\partial z} \Dot{z}
\end{equation}
\end{definition}

\begin{example}
The Chain Rule \\
\textit{Stewart, Calculus: Early Transcendentals, Example 14.5.1} \\
If $f(x, y) = x^2y + 3xy^4$, where $x = \sin 2t$ and $y = \cos t$, find $\frac{df}{dt}$ when $t = 0$. \\

\textbf{Solution:} \\
By the Chain Rule, 
\[ \frac{df}{dt} = (2xy + 3y^4)(2\cos 2t) + (x^2 + 12xy^3)(-\sin t) \]
When $t = 0$, $x = \sin 0 = 0$ and $y = \cos 0 = 1$. Therefore
\[ \frac{df}{dt} \Big|_{t = 0} = (0 + 3)(2) + (0 + 0)(-0) = 6 \]
\end{example}

\subsection{Anti-derivative of Partial Derivatives}

Again, let's review the anti-derivative of a single variable function before moving into higher dimensions. Consider the function
\[ f(x) = x^3 + 2 \]
Whose derivative is 
\[ f'(x) = 3x^2 \]
And if we were to find the anti-derivative of $f'(x)$, 
\[ f(x) = \int f'(x) \ dx = \int 3x^2 \ dx = x^3 + C \]
Since the derivative of any constant is zero, the value $+2$ in the original function is inevitably lost when we attempt to integrate $f'(x)$---which is why we introduce the integration constant $+C$. \\

Consider the multivariable function from the example above,
\[ f(x, y) = x^3 + x^2 y^3 - 2y^2 \]
With partial derivatives
\[ f_x(x, y) = 3x^2 + 2xy^3 \]
\[ f_y(x, y) = 3x^2y^2 - 4y \]
It may be tempting to just say
\[ f(x, y) = \int f_x \ dx \]
But if we explicitly carry out the integration, 
\[ \int f_x \ dx = \int 3x^2 + 2xy^3 \ dx = x^3 + x^2y^3 + C \]
You might notice something is wrong. The $-2y^2$ term is completely gone. Indeed, when we take the partial derivative with respect to $x$, every term that only depends on $y$ is treated as a constant. The "Constant" term really means "Constant with respect to $x$, but can depend on $y$". 

Therefore, if we first integrate $f_x$,
\[ f(x, y) = \int 3x^2 + 2xy^3 \ dx = x^3 + x^2y^3 + g(y) \]
And combine it with the integral of $f_y$, 
\[ f(x, y) = \int 3x^2y^2 - 4y \ dy = x^2y^3 - 2y^2 + h(x) \]
Where $g$ and $h$ are functions yet unknown to us. \\
Equate the expressions above, 
\[ x^3 + x^2y^3 + g(y) = x^2y^3 - 2y^2 + h(x) \]
\[ x^3 + g(y) = - 2y^2 + h(x) \]
We can find that
\[ g(y) = -2y^2 \]
And
\[ h(x) = x^3 \]
Therefore
\[ f(x, y) = x^3 + x^2y^3 - 2y^2 + C \]
Is the anti-derivative of the partial derivatives. Still, we cannot rule out a possible constant of integration $C$. 

\section{Vector Products}

Given two vectors, we can define the Dot Product (which is a scalar quantity), and the Cross Product (which is a vector quantity) between them. Since these two vector products are already extensively discussed in Physics 7A, we will go through them with minimal detail. 

\subsection{Dot Product}

The Dot Product of two vectors is a scalar quantity that measures how much the vectors align with each other. 

\begin{definition}
The Dot Product. \\
For vectors $\Vec{A}$ and $\Vec{B}$, 
\begin{equation}
\Vec{A} = A_x\Hat{x} + A_y\Hat{y} + A_z\Hat{z}
\end{equation}
\begin{equation}
\Vec{B} = B_x\Hat{x} + B_y\Hat{y} + B_z\Hat{z}
\end{equation}

Their dot product is defined as
\begin{equation}
\Vec{A} \cdot \Vec{B} = A_xB_x + A_yB_y + A_zB_z = \sum_i A_iB_i
\end{equation}
Or alternatively, 
\begin{equation}
\Vec{A} \cdot \Vec{B} = AB \cos\theta
\end{equation}
Where $A$, $B$ are the magnitudes of the vectors, and $\theta$ is the angle between the $\Vec{A}$, $\Vec{B}$ vectors. 
\end{definition}

\fig{../discussions/figs/0711/dot.png}{Dot Product}{0.4}{0} 

\pagebreak

\begin{definition}
Properties of the Dot Product. \\
The square of a vector is defined as the dot product of the vector with itself, which equals the square of its magnitude.
\begin{equation}
\Vec{A}^2 = \Vec{A} \cdot \Vec{A} = A^2
\end{equation}

The dot product is commutative and distributive. 
\begin{equation}
\Vec{A} \cdot \Vec{B} = \Vec{B} \cdot \Vec{A}
\end{equation}
\begin{equation}
\Vec{A} \cdot (\Vec{B} + \Vec{C}) = \Vec{A} \cdot \Vec{B} + \Vec{A} \cdot \Vec{C}
\end{equation}

We can also apply the Product Rule when taking the derivative of a dot product. 
\begin{equation}
\frac{d}{dt}(\Vec{A} \cdot \Vec{B}) = \frac{d\Vec{A}}{dt} \cdot \Vec{B} + \Vec{A} \cdot \frac{d\Vec{B}}{dt}
\end{equation}
\end{definition}

\subsection{Matrix Determinant}

The Determinant of a matrix is a scalar value that carries some information about the matrix, such as its invertibility or the span of its columns. Most importantly for us, we will use the matrix determinant as a mathematical tool to carry out cross products. 

\begin{definition}
The Determinant of a $2 \times 2$ Matrix. \\
For a $2 \times 2$ Matrix
\begin{equation}
M = 
\begin{bmatrix}
a & b \\
c & d 
\end{bmatrix}
\end{equation}
We define its determinant as the product of the main diagonal minus the product of the off-diagonal. 
\begin{equation}
\det(M) = \begin{vmatrix}
a & b \\
c & d 
\end{vmatrix} = ad - bc
\end{equation}
\end{definition}

\begin{definition}
The Determinant of a $3 \times 3$ Matrix. \\
For a $3 \times 3$ Matrix
\begin{equation}
M = 
\begin{bmatrix}
a & b & c \\
d & e & f \\
g & h & i
\end{bmatrix}
\end{equation}
We define its determinant in terms of the entries in the first row, and the determinants of the $2 \times 2$ sub-matrices excluding the row and column of the aforementioned entries. 
\begin{equation}
\det(M) = 
a 
\begin{vmatrix}
e & f \\
h & i 
\end{vmatrix}
- b 
\begin{vmatrix}
d & f \\
g & i 
\end{vmatrix}
+c
\begin{vmatrix}
d & e \\
g & h 
\end{vmatrix}
\end{equation}
Expanding the $2 \times 2$ determinants, we have
\begin{equation}
\det(M) = a(ei - fh) - b(di - fg) + c(dh - eg)
\end{equation}
\end{definition}

\begin{example}
The Determinant of a $3 \times 3$ Matrix \\
\textit{Lay, Linear Algebra and Its Applications, Example 3.1.1} \\
Compute the determinant of
\[ A = 
\begin{bmatrix}
1 & 5 & 0 \\
2 & 4 & -1 \\
0 & -2 & 0
\end{bmatrix} \]

\textbf{Solution:} \\
By definition, 
\[ \det(A) = 
1
\begin{vmatrix}
4 & -1 \\
-2 & 0 
\end{vmatrix}
- 5 
\begin{vmatrix}
2 & -1 \\
0 & 0
\end{vmatrix}
+ 0
\begin{vmatrix}
2 & 4 \\
0 & -2
\end{vmatrix}\]
\[ \det(A) = 1(0 - 2) - 5(0 - 0) + 0(-4 - 0) \]
Therefore, 
\[ \det(A) = -2 \]
\end{example}

\subsection{Cross Product}

The Cross Product of two vectors is a vector quantity that is perpendicular to both initial vectors. 

\begin{definition}
The Cross Product. \\
For vectors $\Vec{A}$ and $\Vec{B}$, 
\begin{equation}
\Vec{A} = A_x\Hat{x} + A_y\Hat{y} + A_z\Hat{z}
\end{equation}
\begin{equation}
\Vec{B} = B_x\Hat{x} + B_y\Hat{y} + B_z\Hat{z}
\end{equation}

Their cross product is defined as the $3 \times 3$ determinant
\begin{equation}
\Vec{A} \times \Vec{B} = 
\begin{vmatrix}
\Hat{x} & \Hat{y} & \Hat{z} \\
A_x & A_y & A_z \\
B_x & B_y & B_z
\end{vmatrix}
\end{equation}
Or alternatively, 
\begin{equation}
\Vec{A} \times \Vec{B} = 
\begin{vmatrix}
A_y & A_z \\
B_y & B_z
\end{vmatrix} \Hat{x}
- 
\begin{vmatrix}
A_x & A_z \\
B_x & B_z
\end{vmatrix} \Hat{y}
+ 
\begin{vmatrix}
A_x & A_y \\
B_x & B_y
\end{vmatrix} \Hat{z}
\end{equation}
Expanding the $2 \times 2$ determinants, 
\begin{equation}
\Vec{A} \times \Vec{B} = (A_yB_z - A_zB_y) \Hat{x} - (A_xB_z - A_zB_x) \Hat{y} + (A_xB_y - A_yB_x) \Hat{z}
\end{equation}
\end{definition}

\fig{figs/n0/times.png}{Cross Product and the Right-Hand Rule}{0.5}{0} 

\begin{definition}
Properties of the Cross Product. \\
The cross product, $\Vec{A} \times \Vec{B}$, is perpendicular to both $\Vec{A}$ and $\Vec{B}$. 
\begin{equation}
(\Vec{A} \times \Vec{B}) \perp \Vec{A}
\end{equation}
\begin{equation}
(\Vec{A} \times \Vec{B}) \perp \Vec{B}
\end{equation}

The magnitude of a cross product equals to
\begin{equation}
|\Vec{A} \times \Vec{B}| = AB \sin\theta
\end{equation}
Where $A$, $B$ are the magnitudes of the vectors, and $\theta$ is the angle between the $\Vec{A}$, $\Vec{B}$ vectors. \\

The direction of the resulting vector is given by the Right-Hand Rule, as illustrated in the figure above. To find the direction of $(\Vec{A} \times \Vec{B})$, 
\begin{enumerate}
\item Align the palm of your right hand in the direction of $\Vec{A}$ (The first vector),
\item Bend your four fingers toward the direction of $\Vec{B}$ (The second vector),
\item Then your thumb would naturally point toward the direction of the resultant vector $(\Vec{A} \times \Vec{B})$. 
\end{enumerate}
\textit{If you reverse the order of steps 1 and 2, or if you use the left hand, your answer will end up in the opposite direction. (Sorry lefties!)} \\

The cross product of a vector with itself is the zero vector. In fact, because $\sin(0 \degree) = \sin(180 \degree) = 0$, the cross product of all parallel/antiparallel vectors are zero. 
\begin{equation}
\Vec{A} \times \Vec{A} = \Vec{0}
\end{equation}
The cross product is anti-commutative and distributive. 
\begin{equation}
\Vec{A} \times \Vec{B} = - \Vec{B} \times \Vec{A}
\end{equation}
\begin{equation}
\Vec{A} \times (\Vec{B} + \Vec{C}) = \Vec{A} \times \Vec{B} + \Vec{A} \times \Vec{C}
\end{equation}

We can also apply the Product Rule when taking the derivative of a cross product. 
\begin{equation}
\frac{d}{dt}(\Vec{A} \times \Vec{B}) = \frac{d\Vec{A}}{dt} \times \Vec{B} + \Vec{A} \times \frac{d\Vec{B}}{dt}
\end{equation}
\end{definition}

\subsection{Example: The Triple Product}
\begin{example}
The Triple Product \\
\textit{Taylor, Classical Mechanics, Problem 1.19} \\
If $\Vec{r}, \Vec{v}, \Vec{a}$ denote the position, velocity, and acceleration of a particle, prove that 
\[ \frac{d}{dt}\Bigl[\Vec{a} \cdot (\Vec{v} \times \Vec{r})\Bigr] = \Dot{\Vec{a}} \cdot (\Vec{v} \times \Vec{r}) \]
Where $\Dot{\Vec{a}} = \frac{d\Vec{a}}{dt}$ is the time derivative of acceleration. \\

\textbf{Solution:} Use the Product Rule, 
\[ \frac{d}{dt}\Bigl[\Vec{a} \cdot (\Vec{v} \times \Vec{r})\Bigr] = \frac{d\Vec{a}}{dt} \cdot (\Vec{v} \times \Vec{r}) + \Vec{a} \cdot \frac{d}{dt}(\Vec{v} \times \Vec{r}) \]
Use the product rule again on the last term, 
\[ \frac{d}{dt}\Bigl[\Vec{a} \cdot (\Vec{v} \times \Vec{r})\Bigr] = \frac{d\Vec{a}}{dt} \cdot (\Vec{v} \times \Vec{r}) + \Vec{a} \cdot (\frac{d\Vec{v}}{dt} \times \Vec{r}) + \Vec{a} \cdot (\Vec{v} \times \frac{d\Vec{r}}{dt}) \]
Since $\frac{d\Vec{v}}{dt} = \Vec{a}$ and $\frac{d\Vec{r}}{dt} = \Vec{v}$, 
\[ \frac{d}{dt}\Bigl[\Vec{a} \cdot (\Vec{v} \times \Vec{r})\Bigr] = \frac{d\Vec{a}}{dt} \cdot (\Vec{v} \times \Vec{r}) + \Vec{a} \cdot (\Vec{a} \times \Vec{r}) + \Vec{a} \cdot (\Vec{v} \times \Vec{v}) \]
Since: 
\begin{enumerate}
\item $(\Vec{a} \times \Vec{r})$ is perpendicular to $\Vec{a}$, the dot product $\Vec{a} \cdot (\Vec{a} \times \Vec{r})$ must equal to zero. 
\item $\Vec{v}$ is parallel to $\Vec{v}$, the cross product $\Vec{v} \times \Vec{v}$ must also equal to zero. 
\end{enumerate}
Therefore, 
\[ \frac{d}{dt}\Bigl[\Vec{a} \cdot (\Vec{v} \times \Vec{r})\Bigr] = \frac{d\Vec{a}}{dt} \cdot (\Vec{v} \times \Vec{r}) \]
Or
\[ \frac{d}{dt}\Bigl[\Vec{a} \cdot (\Vec{v} \times \Vec{r})\Bigr] = \Dot{\Vec{a}} \cdot (\Vec{v} \times \Vec{r}) \]
\end{example}

\section{The Differential Operator}

\subsection{Formulation}

Before we can unlock the full potential of the physical theory, we must introduce one more piece of mathematical tool, the Differential Operator (which is also called "Nabla" or "Del"). 

\begin{definition}
The Differential Operator. \\
The Differential Operator $\nabla$ is a vectorial quantity given in Cartesian Coordinates by
\begin{equation}
\nabla = \frac{\partial}{\partial x}\Hat{x} + \frac{\partial}{\partial y}\Hat{y} + \frac{\partial}{\partial z}\Hat{z}
\end{equation}
The Differential Operator, by itself, carries little physical meaning. However, when combined with appropriate mathematical operations, it is widely exploited in all branches of physics. 
\end{definition}

\begin{definition}
The Gradient of a Scalar Function. \\
For a scalar function in three dimensions 
\begin{equation}
f = f(x, y, z)
\end{equation}
Its gradient $\nabla f$ is a vector quantity defined as $\nabla$ times $f$: 
\begin{equation}
\nabla f = \frac{\partial f}{\partial x}\Hat{x} + \frac{\partial f}{\partial y}\Hat{y} + \frac{\partial f}{\partial z}\Hat{z}
\end{equation}
Where each component of $\nabla f$ is the partial derivative of $f$ with respect to that direction. 
\end{definition}

\begin{definition}
The Divergence of a Vector Function. \\
For a vector function in three dimensions
\begin{equation}
\Vec{A} = A_x\Hat{x} + A_y\Hat{y} + A_z\Hat{z}
\end{equation}
Its divergence $\nabla \cdot \Vec{A}$ is a scalar quantity defined as the dot product of $\nabla$ and $\Vec{A}$: 
\begin{equation}
\nabla \cdot \Vec{A} = \frac{\partial A_x}{\partial x} + \frac{\partial A_y}{\partial y} + \frac{\partial A_z}{\partial z}
\end{equation}
Qualitatively, $\nabla \cdot \Vec{A}$ measures the outward flow of the vector $\Vec{A}$ at every point in space. 
\end{definition}

\begin{definition}
The Curl of a Vector Function. \\
For a vector function in three dimensions
\begin{equation}
\Vec{A} = A_x\Hat{x} + A_y\Hat{y} + A_z\Hat{z}
\end{equation}
Its curl $\nabla \times \Vec{A}$ is a vector quantity defined as the cross product of $\nabla$ and $\Vec{A}$: 
\begin{equation}
\nabla \times \Vec{A} = 
\begin{vmatrix}
\Hat{x} & \Hat{y} & \Hat{z} \\
\frac{\partial }{\partial x} & \frac{\partial }{\partial y} & \frac{\partial }{\partial z} \\
A_x & A_y & A_z
\end{vmatrix}
\end{equation}

Or, 
\[ \nabla \times \Vec{A} = \Bigl(\frac{\partial A_z}{\partial y} - \frac{\partial A_y}{\partial z}\Bigr)\Hat{x} - \Bigl(\frac{\partial A_z}{\partial x} - \frac{\partial A_x}{\partial z}\Bigr)\Hat{y} + \Bigl(\frac{\partial A_y}{\partial x} - \frac{\partial A_x}{\partial y}\Bigr)\Hat{z} \]
Qualitatively, $\nabla \times \Vec{A}$ counter-clockwise rotation of the vector $\Vec{A}$ at every point in space. 
\end{definition}

\begin{definition}
The Laplace Operator. \\
The Laplace Operator $\nabla^2$ is defined as the square of the Differential Operator. 
\begin{equation}
\nabla^2 = \nabla \cdot \nabla = \frac{\partial^2}{\partial x^2} + \frac{\partial^2}{\partial y^2} + \frac{\partial^2}{\partial z^2}
\end{equation}
For a scalar function in three dimensions
\begin{equation}
f = f(x, y, z)
\end{equation}
Its Laplacian $\nabla^2 f$ is a scalar quantity
\begin{equation}
\nabla^2 f = \frac{\partial^2f}{\partial x^2} + \frac{\partial^2f}{\partial y^2} + \frac{\partial^2f}{\partial z^2}
\end{equation}
\end{definition}

\subsection{Examples: Gradient, Divergence, and Curl}

\begin{example}
The Gradient of a Scalar Function \\
\textit{Taylor, Classical Mechanics, Problem 4.13} \\
Calculate the gradient $\nabla f$ of
\[ f(r) = \ln(r) \]
Where
\[ r(x, y, z) = \sqrt{x^2 + y^2 + z^2} \]

\textbf{Solution:} Use the Chain Rule, 
\[ \frac{\partial f}{\partial x} = \frac{df}{dr}\frac{\partial r}{\partial x} \]
Therefore, 
\[ \frac{\partial f}{\partial x} = \Bigl(\frac{1}{r}\Bigr)\Bigl(\frac{x}{\sqrt{x^2 + y^2 + z^2}}\Bigr) = \frac{x}{r^2} \]
Similarly,
\[ \frac{\partial f}{\partial y} = \frac{df}{dr}\frac{\partial r}{\partial y} \]
\[ \frac{\partial f}{\partial y} = \Bigl(\frac{1}{r}\Bigr)\Bigl(\frac{y}{\sqrt{x^2 + y^2 + z^2}}\Bigr) = \frac{y}{r^2} \]
And
\[ \frac{\partial f}{\partial z} = \frac{df}{dr}\frac{\partial r}{\partial z} \]
\[ \frac{\partial f}{\partial z} = \Bigl(\frac{1}{r}\Bigr)\Bigl(\frac{z}{\sqrt{x^2 + y^2 + z^2}}\Bigr) = \frac{z}{r^2} \]
Therefore, 
\[ \nabla f = \frac{x}{r^2}\Hat{x} + \frac{y}{r^2}\Hat{y} + \frac{z}{r^2}\Hat{z} \]
\end{example}

\begin{example}
The Curl and Divergence of a Vector Function \\
\textit{Stewart, Calculus: Early Transcendentals, Example 16.5.1} \\
Find the Curl and Divergence of
\[ \Vec{F} = xz \ \Hat{x} + xyz \ \Hat{y} - y^2 \ \Hat{z} \]

\textbf{Solution:} Using the definition of Curl,
\[ \nabla \times \Vec{F} = 
\begin{vmatrix}
\Hat{x} & \Hat{y} & \Hat{z} \\
\frac{\partial }{\partial x} & \frac{\partial }{\partial y} & \frac{\partial }{\partial z} \\
xz & xyz & -y^2
\end{vmatrix}
\]
\[ \nabla \times \Vec{F} = (-2y - xy)\Hat{x} - (0 - x)\Hat{y} + (yz - 0)\Hat{z} \]
The Curl of $\Vec{F}$ is thus
\[ \nabla \times \Vec{F} = (-2y - xy)\Hat{x} + x \ \Hat{y} + yz \ \Hat{z} \]
Using the definition of divergence, 
\[ \nabla \cdot \Vec{F} = \frac{\partial}{\partial x}(xz) + \frac{\partial}{\partial y}(xyz) + \frac{\partial}{\partial z}(-y^2) \]
The Divergence of $\Vec{F}$ is thus
\[ \nabla \cdot \Vec{F} = z + xz \]
\end{example}

\section{Line Integrals}

The work done by a force is given by the integral of the force along a path. 
\[ W = \int_{\Vec{r}_0}^{\Vec{r}_f} \Vec{F} \cdot d\Vec{r} \]
Let us consider the scenario where one drags an object along the table surface. Say the object's trajectory forms a closed path, and it ends up at the same position as where it started. Kinetic friction does work and dissipates energy. 

\fig{figs/n0/path.png}{A Closed Path}{0.65}{0}

If we were to directly plug in ${\Vec{r}_f} = {\Vec{r}_0}$, 
\[ W = \int_{\Vec{r}_0}^{\Vec{r}_0} \Vec{F} \cdot d\Vec{r} \]
We would get that $W = 0$ because the bounds of integration are equal. However, this answer is clearly unphysical. The work should not be zero, as friction does negative work. In this case, we must rewrite $d\Vec{r}$ in terms of a non-identical variable and integrate with respect to that variable. 

Recall that
\[ \frac{d\Vec{r}}{dt} = \Vec{v} \]
Therefore
\[ d\Vec{r} = \Vec{v} \ dt \]
Plug this relationship into the work integral
\[ W = \int_{t_0}^{t_f} \Vec{F} \cdot \Vec{v} \ dt = \oint_C \ \Vec{F} \cdot d\Vec{r} \]
We use the closed integral symbol ($\oint$) to denote integration along a closed path. This agrees with physical intuition: The integrand $\Vec{F} \cdot \Vec{v}$ is the Power delivered by the force, and integrating $\int P \ dt = \Delta E$ indeed gives the change in energy. 

\section{Applications of Vector Calculus}
\subsection{Potential and Conservative Forces}

\begin{definition}
Conservative Forces. \\
The statements below are equivalent. That is, if one statement is satisfied, all other statements must also be true; if one statement is not satisfied, then all other statements must also be false. 
\begin{enumerate}
\item The force $\Vec{F}$ is a Conservative Force. 
\item There exists some scalar function $U$, called the potential energy, such that $\Vec{F}$ can be written as the negative gradient of the potential energy $U$. 
\begin{equation}
\Vec{F} = - \nabla U
\end{equation}
Where
\begin{equation}
\Vec{F} = - \frac{\partial U}{\partial x}\Hat{x} - \frac{\partial U}{\partial y}\Hat{y} - \frac{\partial U}{\partial z}\Hat{z}
\end{equation}
\item The curl of $\Vec{F}$ is the zero vector. 
\begin{equation}
\nabla \times \Vec{F} = \Vec{0}
\end{equation}
\item The work done by the force 
\begin{equation}
W = \int_{\Vec{r}_0}^{\Vec{r}_f} \Vec{F} \cdot d\Vec{r}
\end{equation}
Depends only on the initial and final positions (${\Vec{r}_0}$, ${\Vec{r}_f}$), regardless of the path taken from the initial to the final position. 
\item The work done by the force along a closed loop is zero. 
\begin{equation}
W_C = \oint_C \ \Vec{F} \cdot d\Vec{r} = 0
\end{equation}
\end{enumerate}
\end{definition}

Following the second statement above, if we are given a force $\Vec{F}$, we can find its associated potential energy $U$ from the negative anti-derivative of each component of $\Vec{F}$. 

\begin{definition}
Potential Energy. \\
Given a force $\Vec{F}$, let $F_x$, $F_y$, $F_z$ be the $x$, $y$, and $z$ components of $\Vec{F}$. That is, 
\[ \Vec{F} = F_x\Hat{x} + F_y\Hat{y} + F_z\Hat{z} \]
If $\Vec{F}$ is the conservative force of some potential energy $U$, then
\begin{equation}
F_x = - \frac{\partial U}{\partial x}\Hat{x}
\end{equation}
\begin{equation}
F_y = - \frac{\partial U}{\partial y}\Hat{y}
\end{equation}
\begin{equation}
F_z = - \frac{\partial U}{\partial z}\Hat{z}
\end{equation}
Therefore we can construct the following integrals
\begin{equation}
U_1 = - \int F_x \ dx + f(y, z)
\end{equation}
\begin{equation}
U_2 = - \int F_y \ dy + g(x, z)
\end{equation}
\begin{equation}
U_3 = - \int F_z \ dz + h(x, y)
\end{equation}
For arbitrary functions $f$, $g$, and $h$. 

If some $f$, $g$, and $h$ can be found such that $U_1 = U_2 = U_3 = U$, then $\Vec{F}$ is a conservative force and $U$ is its potential energy. 
\end{definition}

\subsection{Physical Examples}

\begin{example}
Conservative Force and Potential Energy \\
\textit{Giancoli, Physics for Scientists and Engineers, Problem 8.8} \\
Given a force
\[ \Vec{F} = (-6x - 2y)\Hat{x} + (-2x -8yz)\Hat{y} + (-4y^2)\Hat{z} \]
(a) Is the force conservative? \\
(b) If so, what is its potential energy $U$? \\

\textbf{Solution:} \\
(a) Take the curl of $\Vec{F}$, 
\[ \nabla \times \Vec{F} = 
\begin{vmatrix}
\Hat{x} & \Hat{y} & \Hat{z} \\
\frac{\partial }{\partial x} & \frac{\partial }{\partial y} & \frac{\partial }{\partial z} \\
(-6x - 2y) & (-2x -8yz) & -4y^2
\end{vmatrix} \]
\[ \nabla \times \Vec{F} = (-8y + 8y)\Hat{x} - (0 - 0)\Hat{y} + (-2 + 2)\Hat{z} \]
\[ \nabla \times \Vec{F} = \Vec{0} \]
The curl of $\Vec{F}$ is zero. $\Vec{F}$ is conservative. \\

(b) Consider the following integrals: 
\[ U_1 = - \int -6x - 2y \ dx + f(y, z) \]
\[ U_1 = 3x^2 + 2xy + f(y, z) \]
\[ U_2 = - \int -2x -8yz \ dy + g(x, z) \]
\[ U_2 = 2xy + 4y^2 z + g(x, z) \]
\[ U_3 = - \int -4y^2 \ dz + h(x, y) \]
\[ U_3 = 4y^2 z + h(x, y) \]
Equate $U = U_1 = U_2 = U_3$, we can find that
\[ f(y, z) = 4y^2 z \]
\[ g(x, z) = 3x^2 \]
\[ h(x, y) = 3x^2 + 2xy \]
Therefore we can determine potential energy being (up to some constant of integration $C$),
\[ U = 3x^2 + 2xy +4y^2z + C \]
\end{example}

\pagebreak

\begin{example}
Force and Work Done \\
\textit{Taylor, Classical Mechanics, Problem 4.2} \\
Evaluate the work done by a 2-Dimensional Force
\[ \Vec{F} = (x^2)\Hat{x} + (2xy)\Hat{y} \]
Along the path
\[ x = t^3 \]
\[ y = t^2 \]
Joining the origin and the point $(1, 1)$ during time $0 \leq t \leq 1$. \\

\textbf{Solution:} The work done by the force is
\[ W = \int_{(0, 0)}^{(1, 1)} \Vec{F} \cdot d\Vec{r} = \int_0^1 \Vec{F} \cdot \Vec{v} \ dt \]
We rewrite $\Vec{F}$ in terms of time $t$, 
\[ \Vec{F} = (t^6)\Hat{x} + (2t^5)\Hat{y} \]
And the velocity vector is
\[ \Vec{v} = \frac{d\Vec{r}}{dt} = (3t^2) \Hat{x} + (2t) \Hat{y} \]
Therefore
\[ \Vec{F} \cdot \Vec{v} = 3t^8 + 4t^6 \]
And the work done is
\[ W = \int_0^1 3t^8 + 4t^6 \ dt = \Bigl[ \frac{3}{9}t^9 + \frac{4}{7} t^7 \Bigr]_0^1 = \frac{19}{21} \]
\end{example}

These sets of notes are by no means extensive. It is impossible to cover the entire subject of vector calculus in one chapter of notes. However, this should give you a powerful mathematical framework to work with forces and energy in Physics 7A, until you can gain an in-depth understanding of vector calculus from Math 53 (Multivariable Calculus). 

\pagebreak

\section{Exercises}

In the problems below, the variable $\Vec{r}$ refers to the position vector, defined as
\[ \Vec{r} = x\Hat{x} + y\Hat{y} + z\Hat{z} \]
Its magnitude, $r$, equals to
\[ r = \sqrt{x^2 + y^2 + z^2} \]
Its unit vector, $\Hat{r}$, is thus
\[ \Hat{r} = \frac{1}{r}\Vec{r} = \frac{x}{r}\Hat{x} + \frac{y}{r}\Hat{y} + \frac{z}{r}\Hat{z} \]

\begin{problem}
The Divergence of a Point Source. \\
\textit{Griffiths, Introduction to Electrodynamics, Problem 1.16} \\
Find the Divergence, $\nabla \cdot \Vec{F}$, of this vector function. 
\[ \Vec{F} = \frac{1}{r^2}\Hat{r} \]
\end{problem}

\begin{problem}
The Product Rule. \\
\textit{Griffiths, Introduction to Electrodynamics, Problem 1.21} \\
Prove the following vector calculus identity.
\[ \nabla(fg) = f\nabla g + g\nabla f \]
Where $f = f(x, y, z)$ and $g = g(x, y, z)$ are arbitrary functions in 3-Dimensions. 
\end{problem}

\begin{problem}
Gradients. \\
\textit{Taylor, Classical Mechanics, Problem 4.12} \\
For the potential energies below, find the associated conservative force, $\Vec{F} = -\nabla U$. 
\[ U = x^2 + z^3 \]
\[ U = ky \]
\[ U = r \]
\[ U = \frac{1}{r} \]
Where $k$ is a constant. 
\end{problem}

\pagebreak

\begin{problem}
The 3-Dimensional Spring Potential. \\
\textit{Taylor, Classical Mechanics, Problem 4.16} \\
If a particle's potential energy is given by 
\[ U(r) = \frac{1}{2}kr^2 \]
What is the force on the particle? 
\end{problem}

\begin{problem}
Conservative Forces. \\
\textit{Taylor, Classical Mechanics, Problem 4.23} \\
Which of the following forces is conservative? For the forces that are conservative, find their associated potential energy $U$, and verify by taking the gradient that $\Vec{F} = - \nabla U$. \\
\[ \Vec{F} = (kx)\Hat{x} + (2ky)\Hat{y} + (3kz)\Hat{z} \]
\[ \Vec{F} = (ky)\Hat{x} + (kx)\Hat{y} \]
\[ \Vec{F} = (-ky)\Hat{x} + (kx)\Hat{y} \]
Where $k$ is a constant. 
\end{problem}

\end{document}
