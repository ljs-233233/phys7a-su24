\documentclass[11pt]{article}
\usepackage[pdftex]{graphicx}
\usepackage[explicit]{titlesec}
\usepackage[OT1]{fontenc}
\usepackage[most]{tcolorbox}
\usepackage[final]{pdfpages}
\usepackage[colorlinks=true, urlcolor=cyan, hyperfootnotes=false]{hyperref}
\usepackage{fullpage, graphicx, psfrag, url, caption, authblk, amsfonts, amsmath, amssymb, float, fancyhdr, multicol, cmbright, xcolor, amsthm, gensymb, physics}

\fancypagestyle{pages}{
	%Headers
	\fancyhead[L]{Physics 7A, Summer 2024 \\ Section 103}
	%\fancyhead[C]{\thepage}
	\fancyhead[R]{Capstone}
\renewcommand{\headrulewidth}{0pt}
	%Footers
	%\fancyfoot[L]{}
	\fancyfoot[C]{}
	\fancyfoot[R]{\thepage}
\renewcommand{\footrulewidth}{0pt}
}

\newcommand\blfootnote[1]{
    \begingroup
    \renewcommand\thefootnote{}\footnote{#1}
    \addtocounter{footnote}{-1}
    \endgroup
}

\newcommand{\fig}[4]{
    \begin{figure}[H]
        \centering
        \includegraphics[scale={#3}, angle={#4}]{#1}
        \caption{#2}
        \label{exp4fit}
    \end{figure}
}

\newtheoremstyle{gangnamstyle}{}{}{}{}{\sffamily\bfseries}{.}{ }{}
\tcolorboxenvironment{definition}{boxrule=0pt,boxsep=0pt,colback={blue!10},left=8pt,right=8pt,enhanced jigsaw, borderline west={2pt}{0pt}{blue},sharp corners,before skip=10pt,after skip=10pt,breakable}
\tcolorboxenvironment{example}{boxrule=0pt,boxsep=0pt,colback={orange!10},left=8pt,right=8pt,enhanced jigsaw, borderline west={2pt}{0pt}{orange},sharp corners,before skip=10pt,after skip=10pt,breakable}
\tcolorboxenvironment{problem}{boxrule=0pt,boxsep=0pt,colback={cyan!10},left=8pt,right=8pt,enhanced jigsaw, borderline west={2pt}{0pt}{cyan},sharp corners,before skip=10pt,after skip=10pt,breakable}
\theoremstyle{gangnamstyle}{\newtheorem{definition}{Definition}[]}
\theoremstyle{gangnamstyle}{\newtheorem{example}{Example}[]}
\theoremstyle{gangnamstyle}{\newtheorem{problem}{Problem}[]}

\headheight=0pt
\footskip=0pt
\setlength{\oddsidemargin}{0 in}
\setlength{\evensidemargin}{0 in}
\setlength{\topmargin}{-0.5 in}
\setlength{\textwidth}{6.5 in}
\setlength{\textheight}{8.5 in}
\setlength{\headsep}{0.75 in}
\setlength{\parindent}{0 in}
\setlength{\parskip}{0.1 in}

\begin{document}
\normalfont
\pagestyle{pages}

% Begin Document

\begin{center}
\vspace{3in}
{\Large Capstone } \\ [0.05in]
Pendulum with a Moving Support \\ [-0.5in]
\end{center}

\textit{Note: This problem is extremely challenging, and will require you to combine most of the concepts and skills you've learned in this class to formulate a solution. You should give yourself at least an hour to thoroughly think and solve this problem.} 

A bead of mass $m$ is fixed to move on a horizontal frictionless rail, and is connected to another bead of the same mass with a rope of negligible mass and length $L$. The system is subject to a uniform gravitational field $g$. 

Denote the top mass as $m_1$, and the bottom mass as $m_2$. Let $x$ be the horizontal position of $m_1$ along the rail. Let $\theta$ be the angle between the rope and the vertical direction. Assume that the oscillation of the bottom mass is very small $(\theta \approx 0)$, find the oscillation frequency $\omega_{\text{SHM}}$ of the bottom mass, in terms of $L$ and $g$. 

\fig{figs/caps/pendulum.png}{Pendulum with a Moving Support. Not Drawn to Scale.}{0.85}{0}

Since there are two variables in the system ($x$ and $\theta$), we need two independent equations in order to isolate an expression only involving $\theta$. \\
Take the following steps to solve this problem: 

(a) What are the velocities $\Vec{v}_{1}$ and $\Vec{v}_{2}$ of the two masses? \\
\textcolor{blue}{
\textit{Your answers should be vectorial quantities. Recall the relative velocity formula: }
\[ \Vec{v}_2 = \Vec{v}_1 + \Vec{v}_{\text{rel}} \]
\textit{The net velocity $\Vec{v}_2$ is the vector sum of $\Vec{v}_1$ (the reference frame velocity) plus $\Vec{v}_{\text{rel}}$ (the velocity of $m_2$ viewed from the reference frame of $m_1$).}}

(b) What is the total energy in the system? 

(c) Using the Conservation of Energy, find an equation that involves the quantities $x$ and $\theta$, as well as their first-order derivatives $v_x = \frac{dx}{dt}$ and $\omega = \frac{d\theta}{dt}$, and their second-order derivatives $a_x = \frac{d^2x}{dt^2}$ and $\alpha = \frac{d^2\theta}{dt^2}$. \\
\textcolor{blue}{
\textit{A conserved quantity is a constant, therefore its derivative equals to $0$.}}

(d) Draw free-body diagrams for $m_1$, $m_2$, and the whole system consisting of the two masses and the rope. What are the external forces acting on the system? Just giving the directions is sufficient. 

(e) From your answer to part (d), what other physical quantity is conserved? Differentiate that quantity and set it to zero to get another equation of $x$, $\theta$, and their first/second order derivatives. 

(f) Substitute the equation from (e) into the one from (c). You will get an equation purely in terms of $\theta$ and its first/second order derivatives, eliminating all terms depending on $x$. 

(g) Since $\theta \approx 0$, we can make the following approximations to the equation found in (f). \\
\textcolor{blue}{
\textit{1. From Taylor Expansion, $\sin\theta \approx \theta$ and $\cos\theta \approx 1$. \\
2. Since $\theta$ is close to zero, its square $\theta^2$ and higher powers are magnitudes smaller compared to $\theta$, therefore $\theta^2 \ll \theta$. The same applies to its derivatives: $\omega^2 \ll \omega$ and $\alpha^2 \ll \alpha$. Since $\theta$ is already very small, if higher powers of $\theta$ or its derivatives appear in the equation, we can treat them as zero. }}

(h) Now, the equation can be readily written in the standard form of Simple Harmonic Motion: 
\[ \alpha = \frac{d^2\theta}{dt^2} = -\omega_{\text{SHM}}^2\theta \]
What is the frequency of the oscillation, $\omega_{\text{SHM}}$? 

\pagebreak

\begin{center}
(Blank Page)
\end{center}

\pagebreak

\begin{center}
(Blank Page)
\end{center}

\pagebreak

\end{document}
