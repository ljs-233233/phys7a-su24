\documentclass[11pt]{article}
\usepackage[pdftex]{graphicx}
\usepackage[explicit]{titlesec}
\usepackage[OT1]{fontenc}
\usepackage[most]{tcolorbox}
\usepackage[final]{pdfpages}
\usepackage[colorlinks=true, urlcolor=cyan, hyperfootnotes=false]{hyperref}
\usepackage{fullpage, graphicx, psfrag, url, caption, authblk, amsfonts, amsmath, amssymb, float, fancyhdr, multicol, cmbright, xcolor, amsthm, gensymb, physics}

\fancypagestyle{pages}{
	%Headers
	\fancyhead[L]{Physics 7A, Summer 2024 \\ Section 103}
	%\fancyhead[C]{\thepage}
	\fancyhead[R]{Discussion 9 \\ July 11}
\renewcommand{\headrulewidth}{0pt}
	%Footers
	%\fancyfoot[L]{}
	\fancyfoot[C]{}
	\fancyfoot[R]{\thepage}
\renewcommand{\footrulewidth}{0pt}
}

\newcommand\blfootnote[1]{
    \begingroup
    \renewcommand\thefootnote{}\footnote{#1}
    \addtocounter{footnote}{-1}
    \endgroup
}

\newcommand{\fig}[4]{
    \begin{figure}[H]
        \centering
        \includegraphics[scale={#3}, angle={#4}]{#1}
        \caption{#2}
        \label{exp4fit}
    \end{figure}
}

\newtheoremstyle{gangnamstyle}{}{}{}{}{\sffamily\bfseries}{.}{ }{}
\tcolorboxenvironment{definition}{boxrule=0pt,boxsep=0pt,colback={blue!10},left=8pt,right=8pt,enhanced jigsaw, borderline west={2pt}{0pt}{blue},sharp corners,before skip=10pt,after skip=10pt,breakable}
\tcolorboxenvironment{example}{boxrule=0pt,boxsep=0pt,colback={orange!10},left=8pt,right=8pt,enhanced jigsaw, borderline west={2pt}{0pt}{orange},sharp corners,before skip=10pt,after skip=10pt,breakable}
\tcolorboxenvironment{problem}{boxrule=0pt,boxsep=0pt,colback={cyan!10},left=8pt,right=8pt,enhanced jigsaw, borderline west={2pt}{0pt}{cyan},sharp corners,before skip=10pt,after skip=10pt,breakable}
\theoremstyle{gangnamstyle}{\newtheorem{definition}{Definition}[]}
\theoremstyle{gangnamstyle}{\newtheorem{example}{Example}[]}
\theoremstyle{gangnamstyle}{\newtheorem{problem}{Problem}[]}

\headheight=0pt
\footskip=0pt
\setlength{\oddsidemargin}{0 in}
\setlength{\evensidemargin}{0 in}
\setlength{\topmargin}{-0.5 in}
\setlength{\textwidth}{6.5 in}
\setlength{\textheight}{8.5 in}
\setlength{\headsep}{0.75 in}
\setlength{\parindent}{0 in}
\setlength{\parskip}{0.1 in}

\begin{document}
\normalfont
\pagestyle{pages}

% Begin Document

\begin{center}
\vspace{3in}
{\Large Discussion 9 } \\ [0.05in]
Work and Energy \\ [-0.5in]
\end{center}

\section*{Topics}
Vector Dot Product. Work, Kinetic and Potential Energy. Elastic Force. 

\section{Review}

\subsection{Dot Product}

The \textbf{Dot Product} of two vectors is a scalar quantity. It (to some degree) measures how much the two vectors align. 

For vectors $\Vec{A}$ and $\Vec{B}$, 
\[ \Vec{A} = A_x\Hat{x} + A_y\Hat{y} + A_z\Hat{z} \]
\[ \Vec{B} = B_x\Hat{x} + B_y\Hat{y} + B_z\Hat{z} \]

Their dot product is defined as
\[ \Vec{A} \cdot \Vec{B} = A_xB_x + A_yB_y + A_zB_z = \sum_i A_iB_i \]
Or alternatively, 
\[ \Vec{A} \cdot \Vec{B} = AB \cos\theta \]
Where $A$, $B$ are the magnitudes of the vectors, and $\theta$ is the angle between the $\Vec{A}$, $\Vec{B}$ vectors. 

\fig{figs/0711/dot.png}{Vectors and Dot Product}{0.35}{0} 

Since $\cos(0\degree) = 1$ and $\cos(90\degree) = 0$: 

If $\Vec{A}$ and $\Vec{B}$ are parallel, $\Vec{A} \cdot \Vec{B} = AB$; \\
If $\Vec{A}$ and $\Vec{B}$ are perpendicular\footnote{Or you might hear the terms "normal" or "orthogonal", which mean the same thing as "perpendicular".}, $\Vec{A} \cdot \Vec{B} = 0$.

When we "square" a vector, it means the dot product of the vector with itself, which equals the square of its magnitude.
\[ \Vec{A}^2 = \Vec{A} \cdot \Vec{A} = A^2 \]

Therefore, you may sometimes see that a vector's magnitude is written as
\[ A = \sqrt{\Vec{A}^2} \]

The dot product is \textbf{commutative} and \textbf{distributive}. 
\[ \Vec{A} \cdot \Vec{B} = \Vec{B} \cdot \Vec{A} \]
\[ \Vec{A} \cdot (\Vec{B} + \Vec{C}) = \Vec{A} \cdot \Vec{B} + \Vec{A} \cdot \Vec{C} \]

We can also apply the \textbf{Product Rule} when taking the derivative of a dot product. 
\[ \frac{d}{dt}(\Vec{A} \cdot \Vec{B}) = \frac{d\Vec{A}}{dt} \cdot \Vec{B} + \Vec{A} \cdot \frac{d\Vec{B}}{dt} \]

\subsection{Work and Energy}

\textbf{Work} is defined as the energy delivered by a force along the direction of displacement. 
\[ W = \int \Vec{F} \cdot d\Vec{r} = \int F \cos\theta \ dr \]

And if the force is constant, 
\[ W = \Vec{F} \cdot \Delta \Vec{r} = Fr \cos\theta \]

The \textbf{Translational Kinetic Energy} of an object is defined as
\[ K = \frac{1}{2}m\Vec{v}^2 = \frac{1}{2}m(v_x^2 + v_y^2 + v_z^2) \]
Where $m$ is the object's mass, and $\Vec{v}$ is its velocity. 

The \textbf{Work-Energy Theorem} states that the work done on an object equals the change in its kinetic energy. 
\[ W = \Delta K = \frac{1}{2}m(\Vec{v}_f^2 - \Vec{v}_0^2) \]

\pagebreak

\subsection{Elastic Force}

The \textbf{Elastic Force} exerted by a spring is given by Hooke's Law, whose magnitude equals the displacement length from its equilibrium position $(\Vec{x} - \Vec{x}_0)$ multiplied by its spring constant $(k)$. 
\[ \Vec{F}_s = -k(\Vec{x} - \Vec{x}_0) \]
It is a restoring force, hence the presence of a negative sign in the front. When the displacement is positive, the force is in the negative direction, and vice versa. 

Often times, we let the equilibrium position $(\Vec{x}_0)$ be the origin, and write the elastic force as
\[ \Vec{F}_s = -k\Vec{x} \]

\pagebreak

\section{Vector Algebra}

\textbf{1.} \textit{Kleppner and Kolenkow, An Introduction to Mechanics, Problems 1.2, 1.3} \\ 
\textbf{1.1.} Let $\Vec{A} = 3\Hat{x} - 2\Hat{y} + 5\Hat{z}$, and $\Vec{B} = 6\Hat{x} - 7\Hat{y} + 4\Hat{z}$. \\
Find $\Vec{A}^2$, $\Vec{B}^2$, and $(\Vec{A} \cdot \Vec{B})^2$. 

\vspace{2.5 in}

\textbf{1.2.} Find the angle between $\Vec{A} = 3\Hat{x} + \Hat{y} + \Hat{z}$ and $\Vec{B} = -2 \Hat{x} + \Hat{y} + \Hat{z}$. 

\pagebreak

\textbf{2.} \textit{Kleppner and Kolenkow, An Introduction to Mechanics, Problem 1.5} \\
Show that if $|\Vec{A} + \Vec{B}| = |\Vec{A} - \Vec{B}|$, then $\Vec{A}$ and $\Vec{B}$ are perpendicular.

\pagebreak

\textbf{3.} \textit{Taylor, Classical Mechanics, Problem 1.6} 

For the two vectors
\[ \Vec{A} = \Hat{x} + s\Hat{y} \]
\[ \Vec{B} = \Hat{x} - s\Hat{y} \]
What values of $s$ make $\Vec{A}$ and $\Vec{B}$ orthogonal? 

\pagebreak

\textbf{4.} \textit{Taylor, Classical Mechanics, Problem 1.45} 

When we studied Uniform Circular Motion, we know that the velocity (tangential) is perpendicular to acceleration (radial). Now, we can officially prove this result. Show that if $\Vec{v}$ is a function of time and has constant magnitude $v$, then $\Vec{a} = \frac{d\Vec{v}}{dt}$ is orthogonal to $\Vec{v}$. \\
(In fact, this result can be generalized to any vector with constant magnitude and changing direction.)

\textit{Hint: Take the time derivative of $v^2 = \Vec{v}^2$.}

\pagebreak

\textbf{5.} \textit{Kleppner and Kolenkow, An Introduction to Mechanics, Problem 1.8} \\
Let $\Hat{a}$ and $\Hat{b}$ be unit vectors in the $x$-$y$ plane, making angles $\theta$ and $\phi$ above the $x$-axis, respectively. \\
That is, 
\[ \Hat{a} = \cos\theta \Hat{x} + \sin\theta \Hat{y} \]
\[ \Hat{b} = \cos\phi \Hat{x} + \sin\phi \Hat{y} \]
Prove the following trigonometric identities:
\[ \cos(\theta - \phi) = \cos\theta\cos\phi + \sin\theta\sin\phi \]
\[ \cos(\theta + \phi) = \cos\theta\cos\phi - \sin\theta\sin\phi \]
\textit{Hint: \\
For the first identity: What are the magnitudes of $\Hat{a}$ and $\Hat{b}$? What is the angle between $\Hat{a}$ and $\Hat{b}$? \\
For the second identity: What is the unit vector that point at an angle $\phi$ below the $x$-axis? }

\pagebreak

\section{Work and Energy}

\textbf{6.} \textit{Giancoli, Physics for Scientists and Engineers, Problem 7.43} \\
The resistance of a packing material to a sharp object penetrating it is a force proportional to the fourth power of the penetration depth $x$; that is, $\Vec{F} = -bx^4 \Hat{x}$. Calculate the work done to force a sharp object a distance $d$ into the material.

\pagebreak

\textbf{7.} \textit{Giancoli, Physics for Scientists and Engineers, Problem 7.80} \\
The force exerted by an "imperfect" spring compressed by a fairly large amount $x$ from its normal length is given by 
\[ F = -kx - ax^3 - bx^4 \]
How much work must be done to compress this nonlinear spring by an amount $X$, starting from $x = 0$?

\pagebreak

\textbf{8.} \textit{Giancoli, Physics for Scientists and Engineers, Problem 7.79} \\
A varying force is given by 
\[ \Vec{F} = Ae^{-kx}\Hat{x} \]
Where $x$ is the position. $A$ and $k$ are constants that have units of $N$ and $m^{-1}$, respectively. What is the work done when $x$ goes from $0$ to infinity?

\pagebreak

\textbf{9.} \textit{Giancoli, Physics for Scientists and Engineers, Problem 7.96} \\
A small mass $m$ hangs at rest from a vertical rope of length $l$ that is fixed to the ceiling. A force $\Vec{F}$ is applied on the mass, perpendicular to the taut rope at all times, until the rope is oriented at an angle $\theta = \theta_0$ and the mass has been raised by a vertical distance $h$. Assume the force’s magnitude $F$ is adjusted so that the mass moves at constant speed along its curved trajectory. 

Show that the work done by $\Vec{F}$ during this process equals $mgh$ (which is equivalent to the amount of work it takes to slowly lift a mass $m$ straight up by a height $h$). \\
\textit{Hint: The arc length formula is given by $ds = l d\theta$.}
\fig{figs/0711/giancoli96.png}{Giancoli, Problem 7.96}{0.5}{0} 

\pagebreak

\section{Elastic Force}

\textbf{10.} \textit{Kleppner and Kolenkow, An Introduction to Mechanics, Problem 3.19} \\
Find the effective spring constant $k_{eff}$ of mass $m$ suspended by two springs having constants $k_1$ and $k_2$, in each of the configurations (a) and (b). The effective spring constant is defined such that
\[ F_s = -k_{eff}x_{tot} \]
Where $F_s$ is the elastic force on the mass, and $x_{tot}$ is the total displacement of the mass. The springs are ideal and massless. 
\fig{figs/0711/kk19.png}{Kleppner and Kolenkow, Problem 3.19}{0.5}{0} 
\textit{P.S. If you have studied electromagnetism in high school physics, you might recognize your results are identical to the effective capacitance of capacitors in series/parallel. Indeed, springs in mechanical systems are analogous to capacitors in electrical systems; masses are analogous to inductors; and friction forces, if present, are analogous to resistors.}

\pagebreak

\textbf{11.} \textit{Giancoli, Physics for Scientists and Engineers, Problem 7.99} \\
Stretchable ropes are used to safely arrest the fall of rock climbers. Suppose one end of a rope with unstretched length $l$ is anchored to a cliff and a climber of mass $m$ is attached to the other end. When the climber is a height $l$ above the anchor point, he slips and falls under the force of gravity for a distance $2l$, after which the rope becomes taut and stretches a distance $x$ as it stops the climber. Assume a stretchy rope behaves as a spring with spring constant $k$. 

\fig{figs/0711/giancoli99.png}{Giancoli, Problem 7.99}{0.4}{0} 

Applying the work-energy theorem, show that
\[ x = \frac{mg}{k}\Biggl[ 1 + \sqrt{1 + \frac{4kl}{mg}} \ \Biggr] \]

\end{document}
