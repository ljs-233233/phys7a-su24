\documentclass[11pt]{article}
\usepackage[pdftex]{graphicx}
\usepackage[explicit]{titlesec}
\usepackage[OT1]{fontenc}
\usepackage[most]{tcolorbox}
\usepackage[final]{pdfpages}
\usepackage[colorlinks=true, urlcolor=cyan, hyperfootnotes=false]{hyperref}
\usepackage{fullpage, graphicx, psfrag, url, caption, authblk, amsfonts, amsmath, amssymb, float, fancyhdr, multicol, cmbright, xcolor, amsthm, gensymb, physics}

\fancypagestyle{pages}{
	%Headers
	\fancyhead[L]{Physics 7A, Summer 2024 \\ Section 103}
	%\fancyhead[C]{\thepage}
	\fancyhead[R]{Discussion 19 \\ August 6}
\renewcommand{\headrulewidth}{0pt}
	%Footers
	%\fancyfoot[L]{}
	\fancyfoot[C]{}
	\fancyfoot[R]{\thepage}
\renewcommand{\footrulewidth}{0pt}
}

\newcommand\blfootnote[1]{
    \begingroup
    \renewcommand\thefootnote{}\footnote{#1}
    \addtocounter{footnote}{-1}
    \endgroup
}

\newcommand{\fig}[4]{
    \begin{figure}[H]
        \centering
        \includegraphics[scale={#3}, angle={#4}]{#1}
        \caption{#2}
        \label{exp4fit}
    \end{figure}
}

\newtheoremstyle{gangnamstyle}{}{}{}{}{\sffamily\bfseries}{.}{ }{}
\tcolorboxenvironment{definition}{boxrule=0pt,boxsep=0pt,colback={blue!10},left=8pt,right=8pt,enhanced jigsaw, borderline west={2pt}{0pt}{blue},sharp corners,before skip=10pt,after skip=10pt,breakable}
\tcolorboxenvironment{example}{boxrule=0pt,boxsep=0pt,colback={orange!10},left=8pt,right=8pt,enhanced jigsaw, borderline west={2pt}{0pt}{orange},sharp corners,before skip=10pt,after skip=10pt,breakable}
\tcolorboxenvironment{problem}{boxrule=0pt,boxsep=0pt,colback={cyan!10},left=8pt,right=8pt,enhanced jigsaw, borderline west={2pt}{0pt}{cyan},sharp corners,before skip=10pt,after skip=10pt,breakable}
\theoremstyle{gangnamstyle}{\newtheorem{definition}{Definition}[]}
\theoremstyle{gangnamstyle}{\newtheorem{example}{Example}[]}
\theoremstyle{gangnamstyle}{\newtheorem{problem}{Problem}[]}

\headheight=0pt
\footskip=0pt
\setlength{\oddsidemargin}{0 in}
\setlength{\evensidemargin}{0 in}
\setlength{\topmargin}{-0.5 in}
\setlength{\textwidth}{6.5 in}
\setlength{\textheight}{8.5 in}
\setlength{\headsep}{0.75 in}
\setlength{\parindent}{0 in}
\setlength{\parskip}{0.1 in}

\begin{document}
\normalfont
\pagestyle{pages}

% Begin Document

\begin{center}
\vspace{3in}
{\Large Discussion 19 } \\ [0.05in]
Final Review \\ [-0.5in]
\end{center}

\section{Review}

\textbf{1.} \textit{Past Exam Problem} \\
An adventurer decides to cross the ocean in a cylindrically shaped floating vehicle. The air pressure inside is maintained at $1.3$ atm in absolute pressure, and the total mass of the cylinder, the adventurer and her equipment, and a ballast weight hanging below is $M$. The cylinder has radius $R$; you may neglect the volume of the ballast weight hanging below. 
\fig{figs/0806/1.png}{Pressurized Cylinder}{0.6}{0}

a) What is the depth $D$ of the bottom of the cylinder relative to the surface of the water? Express your answer in terms of $M$, $R$, and the density of water $\rho$.

b) The adventurer discovers a leak ― a fountain of water is shooting up from a small hole in the floor of the cylinder. How high $h$ above the floor does the fountain get before the drops of water fall back downward?

c) The adventurer holds her foot just over the hole so that the rushing water hits the bottom of her boot. If the hole has radius $r$, then what is the force on her boot due to the water? Assume that the water bounces off of her boot elastically.

d) The adventurer grabs a hose and connects it to the hole. The hose has a variable diameter so that it has cross sectional area $A_1$ on one end and cross-sectional area $A_2$ on the other. In this configuration, if water enters the first end at speed $v_1$, then what is the speed of water passing through the other end? Express your answer in terms of $A_1$, $A_2$, and $v_1$.

\pagebreak

\begin{center}
(Blank Page)
\end{center}

\pagebreak

\textbf{2.} \textit{Past Exam Problem} \\
A cell phone tower transmits electromagnetic waves with vacuum wavelength $\lambda_0$. The wave is described by the equation
\[ E(r, t) = A\cos(kr - \omega t + \phi) \]
A person with a cell phone stands a horizontal distance $d$ away. The source of EM waves is a point at the top of the tower, a height H above the ground. The person holds the phone the same height $H$
above the ground. The EM waves can reach the phone either via a straight, direct path or via a path that reflects once off the ground.

This problem came from an electromagnetism class, so here's the additional information you'll need. When an electromagnetic wave reflects off a surface, its phase is increased by $\pi$, or $\cos(x + \pi) = -\cos(x)$, this property is also exhibited by sound waves. 

\fig{figs/0806/2.png}{Cell Phone Signals}{0.6}{0}

(a) Find the path length difference between the two possible paths.

(b) Find the phase difference between the two possible paths.

(c) Find the horizontal distances away from the tower where the person holding the phone would experience completely constructive interference.

(d) Find the horizontal distances away from the tower where the person holding the phone would experience completely destructive interference.

(e) Find the maximum distance, beyond which completely destructive interference is impossible. \\
\textit{Hint: Your solutions in part (d) should correspond to positive integers. Find the smallest integer, below which the distances in part (d) are no longer positive. Show that for integers larger than this minimum integer, the distances in part (d) decrease with increasing integers. Hence conclude that the distance corresponding to this minimum integer is the maximum possible distance for completely destructive interference.}

\pagebreak

\begin{center}
(Blank Page)
\end{center}

\pagebreak

\textbf{3.} \textit{Past Exam Problem} \\
Three rigid, cubical blocks, each of mass $m$ and horizontal length $L$, are connected in sequence by two ideal, massless, horizontal springs, each of spring constant $k$ and un-stretched length $L$. Then the middle mass is cut exactly in half vertically, and the two sub-systems are temporarily separated, only to have the left one launched toward the right one. Before the collision, both connected masses on the left are moving rightward, with the adjoining spring remaining un-stretched, while both connected masses on the right are initially motionless, with the adjoining spring also un-stretched.
\fig{figs/0806/charman.jpg}{Springs to Mind}{0.5}{0}
All masses slide without appreciable friction on a long, immovable, horizontal surface. Any drag forces or internal damping or dissipation in the springs remain negligible. When the split masses suddenly collide, the collision is impulsive, and subsequently they stick together permanently. 

Immediately after the collision: \\
(a) What are the horizontal velocities of (i) the left mass, (ii) the recombined center mass, and (iii) the right mass? 

During the subsequent motion after the recombination: \\
(b) 1. What fraction of the pre-collision, bulk kinetic energy has been lost? \\
2. What fraction resides in the motion of the center-of-mass of the recombined mass-spring system? \\
3. What fraction is associated with the motion of the rest of the system about the center-of-mass?

(c) At what frequency is the system oscillating? 

(d) What are the minimum and maximum speeds of the central mass?

(e) How much time elapses from the time of collision until the earliest subsequent moment when the left mass is instantaneously at rest?

\textit{Hint: You should solve parts (b) - (e) in a convenient frame of reference.}

\pagebreak

\begin{center}
(Blank Page)
\end{center}

\pagebreak

\end{document}
