\documentclass[11pt]{article}
\usepackage[pdftex]{graphicx}
\usepackage[explicit]{titlesec}
\usepackage[OT1]{fontenc}
\usepackage[most]{tcolorbox}
\usepackage[final]{pdfpages}
\usepackage[colorlinks=true, urlcolor=cyan, hyperfootnotes=false]{hyperref}
\usepackage{fullpage, graphicx, psfrag, url, caption, authblk, amsfonts, amsmath, amssymb, float, fancyhdr, multicol, cmbright, xcolor, amsthm, gensymb, physics}

\fancypagestyle{pages}{
	%Headers
	\fancyhead[L]{Physics 7A, Summer 2024 \\ Section 103}
	%\fancyhead[C]{\thepage}
	\fancyhead[R]{Capstone Solution}
\renewcommand{\headrulewidth}{0pt}
	%Footers
	%\fancyfoot[L]{}
	\fancyfoot[C]{}
	\fancyfoot[R]{\thepage}
\renewcommand{\footrulewidth}{0pt}
}

\newcommand\blfootnote[1]{
    \begingroup
    \renewcommand\thefootnote{}\footnote{#1}
    \addtocounter{footnote}{-1}
    \endgroup
}

\newcommand{\fig}[4]{
    \begin{figure}[H]
        \centering
        \includegraphics[scale={#3}, angle={#4}]{#1}
        \caption{#2}
        \label{exp4fit}
    \end{figure}
}

\newtheoremstyle{gangnamstyle}{}{}{}{}{\sffamily\bfseries}{.}{ }{}
\tcolorboxenvironment{definition}{boxrule=0pt,boxsep=0pt,colback={blue!10},left=8pt,right=8pt,enhanced jigsaw, borderline west={2pt}{0pt}{blue},sharp corners,before skip=10pt,after skip=10pt,breakable}
\tcolorboxenvironment{example}{boxrule=0pt,boxsep=0pt,colback={orange!10},left=8pt,right=8pt,enhanced jigsaw, borderline west={2pt}{0pt}{orange},sharp corners,before skip=10pt,after skip=10pt,breakable}
\tcolorboxenvironment{problem}{boxrule=0pt,boxsep=0pt,colback={cyan!10},left=8pt,right=8pt,enhanced jigsaw, borderline west={2pt}{0pt}{cyan},sharp corners,before skip=10pt,after skip=10pt,breakable}
\theoremstyle{gangnamstyle}{\newtheorem{definition}{Definition}[]}
\theoremstyle{gangnamstyle}{\newtheorem{example}{Example}[]}
\theoremstyle{gangnamstyle}{\newtheorem{problem}{Problem}[]}

\headheight=0pt
\footskip=0pt
\setlength{\oddsidemargin}{0 in}
\setlength{\evensidemargin}{0 in}
\setlength{\topmargin}{-0.5 in}
\setlength{\textwidth}{6.5 in}
\setlength{\textheight}{8.5 in}
\setlength{\headsep}{0.75 in}
\setlength{\parindent}{0 in}
\setlength{\parskip}{0.1 in}

\begin{document}
\normalfont
\pagestyle{pages}

% Begin Document

\begin{center}
\vspace{3in}
{\Large Capstone Solution } \\ [0.05in]
Pendulum with a Moving Support \\ [-0.5in]
\end{center}

\textit{Note: This problem is extremely challenging, and will require you to combine most of the concepts and skills you've learned in this class to formulate a solution. You should give yourself at least an hour to thoroughly think and solve this problem.} 

A bead of mass $m$ is fixed to move on a horizontal frictionless rail, and is connected to another bead of the same mass with a rope of negligible mass and length $L$. The system is subject to a uniform gravitational field $g$. 

Denote the top mass as $m_1$, and the bottom mass as $m_2$. Let $x$ be the horizontal position of $m_1$ along the rail. Let $\theta$ be the angle between the rope and the vertical direction. Assume that the oscillation of the bottom mass is very small $(\theta \approx 0)$, find the oscillation frequency $\omega_{\text{SHM}}$ of the bottom mass, in terms of $L$ and $g$. 

\fig{figs/caps/pendulum.png}{Pendulum with a Moving Support. Not Drawn to Scale.}{0.85}{0}

Since there are two variables in the system ($x$ and $\theta$), we need two independent equations in order to isolate an expression only involving $\theta$. \\
Take the following steps to solve this problem: 

\textcolor{blue}{The solution is written in blue. }

\pagebreak

(a) What are the velocities $\Vec{v}_{1}$ and $\Vec{v}_{2}$ of the two masses? 

\textcolor{blue}{In Cartesian Coordinates, let $+x$ be toward the right, and $+y$ be upward. Let $y = 0$ at the height of the rail. }

\textcolor{blue}{Note that the top mass ($m_1$) is free to slide left and right along the horizontal rail, but does not have any vertical motion. The bottom mass ($m_2$) is constrained to move in a circle with radius $L$ with respect to the current position of $m_1$. }

\fig{figs/caps/geometry.jpeg}{(Left) The Pendulum; (Right) The relative motion between $m_{1, 2}$}{0.11}{0}

\textcolor{blue}{From the right figure, the position of $m_2$ relative to $m_1$ is
\[ \Vec{r}_{\text{rel}} = L\sin(\theta) \ \Hat{x} - L\cos(\theta) \ \Hat{y} \]
In terms of the variables $x$ (horizontal position of $m_1$) and $\theta$ (angle between the rope and the vertical), the position of $m_1$ is
\[ \Vec{r}_1 = x \ \Hat{x} \]
And the position of $m_2$ is
\[ \Vec{r}_2 = \Vec{r}_1 + \Vec{r}_{\text{rel}} = [x + L\sin(\theta)] \ \Hat{x} - L\cos(\theta) \ \Hat{y} \]
The velocities of $m_1$ and $m_2$ are the time derivatives of the respective positions.
\begin{equation}
\Vec{v}_1 = v_x \ \Hat{x}
\end{equation}
\begin{equation}
\Vec{v}_2 = [v_x + L\cos(\theta)\omega] \ \Hat{x} + L\sin(\theta)\omega \ \Hat{y}
\end{equation}
Where
\[ v_x = \frac{dx}{dt} \]
\[ \omega = \frac{d\theta}{dt} \]}

\pagebreak

(b) What is the total energy in the system? 

\textcolor{blue}{The total energy in the system is the sum of all kinetic energies and the gravitational potential energy of $m_2$. 
\[ E = K_1 + K_2 + U_2 \]
Recall that the square magnitude of a vector can be written as the dot product with itself: 
\[ v^2 = \Vec{v}^2 = \Vec{v} \cdot \Vec{v} \]
The total energy is thus
\[ E = \frac{1}{2}m\Vec{v}_1^2 + \frac{1}{2}m\Vec{v}_2^2 - mgy_2 \]
\[ E = \frac{1}{2}mv_x^2 + \frac{1}{2}m\Bigl[(v_x + L\cos(\theta)\omega)^2 + (L\sin(\theta)\omega)^2 \Bigr] - mgL\cos(\theta) \]
Simplify this expression, 
\[ E = \frac{1}{2}mv_x^2 + \frac{1}{2}m\Bigl[v_x^2 + 2L\cos(\theta)v_x\omega + L^2\cos^2(\theta)\omega^2 + L^2\sin^2(\theta)\omega^2 \Bigr] - mgL\cos(\theta) \]
\begin{equation}
E = \frac{1}{2}m\Bigl[2v_x^2 + 2L\cos(\theta)v_x\omega + L^2\omega^2 \Bigr] - mgL\cos(\theta)
\end{equation}
} 

(c) Using the Conservation of Energy, find an equation that involves the quantities $x$ and $\theta$, as well as their first-order derivatives $v_x = \frac{dx}{dt}$ and $\omega = \frac{d\theta}{dt}$, and their second-order derivatives $a_x = \frac{d^2x}{dt^2}$ and $\alpha = \frac{d^2\theta}{dt^2}$. 

\textcolor{blue}{
Take the time derivative of $E$, 
\[ 0 = \frac{dE}{dt} = \frac{1}{2}m\Bigl[ 4v_xa_x - 2L\sin(\theta)v_x\omega^2 + 2L\cos(\theta)a_x\omega + 2L\cos(\theta)v_x\alpha + 2L^2\omega\alpha \Bigr] + mgL\sin(\theta)\omega \]
Where
\[ a_x = \frac{dv_x}{dt} \]
And
\[ \alpha = \frac{d\omega}{dt} \]
Simplify the expression above, 
\begin{equation}
0 = 2v_xa_x - L\sin(\theta)v_x\omega^2 + L\cos(\theta)a_x\omega + L\cos(\theta)v_x\alpha + L^2\omega\alpha + gL\sin(\theta)\omega
\end{equation}}

\pagebreak

(d) Draw free-body diagrams for $m_1$, $m_2$, and the whole system consisting of the two masses and the rope. What are the external forces acting on the system? Just giving the directions is sufficient. 

\textcolor{blue}{See below.}

\fig{figs/caps/fbd.jpeg}{Free-Body Diagrams of (Left) $m_1$; (Center) $m_2$; (Right) The Entire System}{0.125}{0}

\textcolor{blue}{Most of the forces here should be obvious. However, if you are confused of why $\Vec{F}_N$ points upward, we present two explanations:}

\textcolor{blue}{1. The normal force is perpendicular to the surface (railing) that $m_1$ moves along.}

\textcolor{blue}{2. Normal force cannot do any work, therefore it must be perpendicular to the velocity of $m_1$, which is purely horizontal.}

(e) From your answer to part (d), what other physical quantity is conserved? Differentiate that quantity and set it to zero to get another equation of $x$, $\theta$, and their first/second order derivatives. 

\textcolor{blue}{From the free-body diagram of the entire system, we observe that all external forces are in the vertical direction. Therefore, the $x$-component of momentum ($p_x$) is conserved. 
\[ p_x = m_1v_{1x} + m_2v_{2x} = mv_x + m(v_x + L\cos(\theta)\omega) \]
\[ p_x = 2mv_x + mL\cos(\theta)\omega \]
The time derivative of $p_x$ is:
\[ 0 = \frac{dp_x}{dt} = 2ma_x - mL\sin(\theta)\omega^2 + mL\cos(\theta)\alpha \]
Rearrange the terms,
\begin{equation}
a_x = \frac{1}{2}\Bigl[ L\sin(\theta)\omega^2 - L\cos(\theta)\alpha \Bigr]
\end{equation}}

\pagebreak

(f) Substitute the equation from (e) into the one from (c). You will get an equation purely in terms of $\theta$ and its first/second order derivatives, eliminating all terms depending on $x$. 

\textcolor{blue}{Plug the expression for $a_x$ into the equation from (c),
\[ \begin{aligned} 0 &= 2v_x\frac{1}{2}\Bigl[ L\sin(\theta)\omega^2 - L\cos(\theta)\alpha \Bigr] - L\sin(\theta)v_x\omega^2 \\
&+ L\cos(\theta)\omega\frac{1}{2}\Bigl[ L\sin(\theta)\omega^2 - L\cos(\theta)\alpha \Bigr] + L\cos(\theta)v_x\alpha + L^2\omega\alpha + gL\sin(\theta)\omega
\end{aligned} \]
Distribute all terms,
\[ \begin{aligned} 0 &= L\sin(\theta)v_x\omega^2 - L\cos(\theta)v_x\alpha - L\sin(\theta)v_x\omega^2 + \frac{1}{2}L^2\cos(\theta)\sin(\theta)\omega^3 \\ 
&- \frac{1}{2}L^2\cos^2(\theta)\omega \alpha + L\cos(\theta)v_x\alpha + L^2\omega\alpha + gL\sin(\theta)\omega
\end{aligned} \]
Note that the first term cancels with the third, and the second term cancels with the sixth, therefore eliminating all terms containing $v_x$. 
\[ 0 = \frac{1}{2}L^2\cos(\theta)\sin(\theta)\omega^3 - \frac{1}{2}L^2\cos^2(\theta)\omega \alpha + L^2\omega\alpha + gL\sin(\theta)\omega \]
Further simplify by eliminating the common factor $\omega$, 
\begin{equation}
0 = \frac{1}{2}L^2\cos(\theta)\sin(\theta)\omega^2 - \frac{1}{2}L^2\cos^2(\theta)\alpha + L^2\alpha + gL\sin(\theta)
\end{equation}
}

(g) Since $\theta \approx 0$, we can make the following approximations to the equation found in (f). 

\textcolor{blue}{
Under the small amplitude assumption, we make these approximations to our answer
\[ \sin\theta \approx \theta \]
\[ \cos\theta \approx 1 \]
\[ \omega^2 \approx 0 \]
And the equation of motion becomes
\[ 0 = 0 - \frac{1}{2}L^2\alpha + L^2\alpha + gL\theta \]
Or
\begin{equation}
0 = \frac{1}{2}L\alpha + g\theta
\end{equation}}

\pagebreak

(h) Now, the equation can be readily written in the standard form of Simple Harmonic Motion: 
\[ \alpha = \frac{d^2\theta}{dt^2} = -\omega_{\text{SHM}}^2\theta \]
What is the frequency of the oscillation, $\omega_{\text{SHM}}$? 

\textcolor{blue}{Putting our answer into the standard form, 
\[ \alpha = \frac{d^2\theta}{dt^2} = - \frac{2g}{L}\theta \]
Therefore the frequency of oscillation is
\[ \omega_{\text{SHM}} = \sqrt{\frac{2g}{L}} \]}

\end{document}
